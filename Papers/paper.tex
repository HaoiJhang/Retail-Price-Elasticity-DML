\documentclass[10pt,a4paper,preprint,authoryear]{elsarticle}
\usepackage{sty-sine/sine}
\usepackage{minted}
\usepackage{xcolor} % 用于配置颜色(可选)
\usemintedstyle{vs} 
\setminted{
    fontsize=\footnotesize,  % 字号
    frame=single,          % 单线边框
    linenos=true,          % 显示行号
    breaklines=true,       % 自动折行
}



\title{需求价格弹性的因果计算}

\author{张浩怡}

\address{经济学院, 2023111228}

\begin{document}

\begin{abstract}

需求价格弹性是企业优化定价策略与提升营收的核心决策依据. 尽管 A/B 测试是理想的评估手段, 但在零售场景中易损害用户体验. 因此, 基于观测数据的因果推断成为主流方案, 但如何剥离季节性、产品异质性等高维混杂因素是其核心难点. 

本研究提出了一种基于双重机器学习 (DML) 的稳健因果推断框架. 我们构建了``随机森林 + Poisson 回归''的混合残差模型, 以捕捉价格形成的非线性机制及销量的离散计数特征. 通过蒙特卡洛模拟与理论推导, 本研究深入剖析了四种主流估计量的偏差来源: (1) 原始 OLS 因忽略商品异质性与供需联立性, 表现出严重的正向偏差;(2) 去均值 (De-meaned) 模型虽剔除了固定效应, 但仍受时变混杂因素干扰;(3) 朴素 DML (Naive DML) 受第一阶段正则化噪音影响, 面临严重的衰减偏差. 相比之下, 本研究采用的 Robust DML 基于 Neyman 正交化构造, 有效抵消了滋扰参数的估计误差, 实现了无偏估计. 

基于 Kaggle 零售数据集的实证结果验证了上述理论: DML 框架将分箱均方根误差 (RMSE) 降低了约 57\%, 极大提升了拟合稳健性. 最终估算的市场平均弹性为 -1.89, 修正了传统模型低估价格敏感度的问题. 本研究为企业在复杂动态环境下实现“智能定价”提供了一套具备理论纠偏能力的算法流程. 

\end{abstract}

\begin{keyword}
价格弹性 \sep 因果推断 \sep 偏差分析 \sep 双重机器学习
\end{keyword}

\maketitle


\section{引言}

在微观经济学与现代商业决策中, 需求价格弹性是衡量市场反应的核心指标. 它反映了消费者需求量对价格变动的敏感程度, 直接决定了企业的定价策略与盈利能力. 从管理决策的角度来看, 准确估计价格弹性具有重要的指导意义: 当弹性绝对值小于 $1$ 时, 适度提价能够增加总收入; 而当弹性绝对值大于 $1$ 时, 降价促销则成为获取市场份额、提升营收的有效手段\citep{nicholson2011intermediate}. 因此, 在动态竞争的市场环境中, 如何精准捕捉需求曲线的斜率, 成为厂商实现收益最大化的关键.

尽管价格弹性在理论上定义明确, 但在实证估算中却面临严峻挑战. 理想的评估手段是进行随机对照试验, 即通过随机向不同用户展示差异化价格来观察反馈. 然而在现实商业环境中, 这种定价策略往往由于可能损害用户体验、导致价格歧视争议及削弱品牌信誉而难以大规模推行. 

因此, 基于历史观测数据进行因果推断成为更为可行的替代方案. 然而, 观测数据并非来自随机分配, 价格与需求之间往往存在复杂的内生性问题. 例如, 季节性波动、促销活动的周期性、以及产品质量的变化等因素, 既会影响企业的定价行为, 也会直接干扰消费者的购买决策. 如果不能有效剥离这些混杂因素的影响, 传统的统计回归模型往往会产生严重偏差, 得出虚假的相关性而非真实的因果效应. 

针对上述难题, 本研究引入了前沿的双重机器学习框架\citep{hua2021markdownsecommercefreshretail}, 旨在从高维、非线性的历史交易数据中提取无偏的价格弹性估值. 相比于传统的计量经济学模型, 本方法在以下两个方面具有显著优势: 
\begin{itemize}
    \item 高维变量的处理能力:  本研究利用正则化技术与特征工程, 从商品代码、日期特征、文本描述及区域分布等海量信息中自动筛选重要控制变量, 有效解决了因变量过多导致的过拟合问题. 
    \item 非线性因果建模:  传统的线性模型难以捕捉价格形成的复杂机制. 本研究在 DML 框架下, 结合了随机森林捕捉非线性交互的能力, 以及 Poisson 回归处理离散销量数据的统计优势, 实现了更为精准的预测与正交化处理. 
\end{itemize}

本文利用 Kaggle 公开的真实零售交易数据集进行实证分析. 实验结果表明, 通过 DML 框架提取的正交化残差能够更清晰地还原需求曲线的线性结构, 显著降低了估计误差, 并为企业在复杂环境下进行``智能定价''提供了稳健的量化支持. 

本文的后续章节结构安排如下:

第 \ref{sec:theory} 节构建了需求价格弹性的理论计量框架,通过数学推导深入剖析了普通最小二乘法 (OLS)、去均值模型 (De-meaned) 及朴素双重机器学习 (Naive DML) 在因果推断中产生偏差的内在机制,并从理论上论证了 Robust DML 估计量的无偏性.

第 \ref{sec:simulation} 节设计了受控的蒙特卡洛模拟实验,在已知真实弹性参数与人为注入噪音的前提下,验证了上述理论推导的正确性,并评估了不同估计量在有限样本下的稳健性.

第 \ref{sec:dataset} 节介绍了实证研究的数据基础与模型设定. 本节详细阐述了 Kaggle 零售数据的预处理流程、基于 NLP 与时间序列的高维特征工程,以及``随机森林 + Poisson 回归''混合模型的具体实施细节. 

第 \ref{sec:evidence} 节展示了实证估算结果与模型诊断,通过分箱回归图重构了需求曲线,并对弹性系数进行了深度分析. 最后,第 \ref{sec:conclusion} 节总结全文, 提出了管理启示与未来展望. 


\section{理论框架}
\label{sec:theory}

为了从理论上厘清不同估计策略的有效性边界, 本节建立了一个包含高维混杂因素的线性需求结构方程模型. 

基于该模型, 本节将首先推导普通最小二乘法及固定效应模型的渐近偏差形式, 揭示``供需联立性''与``遗漏变量''如何扭曲弹性估计. 随后, 我们将重点讨论双重机器学习 框架下的两种估计策略, 从数学上证明朴素 DML在有限样本下的衰减偏差, 并推导基于 Neyman 正交化分数的 Robust DML 估计量如何实现对滋扰参数误差的稳健, 从而获得一致无偏的因果推断结果. 

\subsection{模型设定}

为构建理论分析框架, 假设真实的数据生成过程服从如下线性需求函数:
\begin{align}
    y = \theta x + \alpha_i + \beta S_{i,t} + \varepsilon_{i,t} \label{eq:real_demand_f}
\end{align}

式 \eqref{eq:real_demand_f} 中各变量的经济学含义设定如下:
\begin{itemize}
    \item $y$: 需求量 $Q$ 的对数形式;
    \item  $x$: 价格 $P$ 的对数形式;
    \item $\theta < 0$: 待估计的真实价格弹性;
    \item $\alpha_i$: 商品层面的固定效应 (如未被观测到的商品质量). 通常高质量商品价格较高, 故假设 $\cov{x,\alpha} > 0$. 
    \item $S_{i,t}$: 随时间变化的混杂因素 (如双十一促销、换季清仓等环境因素). 假设此类因素会提升需求 ($\beta > 0$), 且往往伴随着商家的降价行为 ($\cov{x, S} < 0$). 
    \item $\varepsilon_{i,t}$: 独立同分布的随机误差项.
\end{itemize}

\subsection{估计量分析}

\subsubsection{普通最小二乘法 (Raw OLS)}

若忽略潜在的混杂因素, 直接使用普通最小二乘法拟合单变量回归 $y = \theta_{\text{OLS}} x$, 其估计量推导如式 \eqref{eq:raw_ols} :
\begin{align} \label{eq:raw_ols}
    \est{\theta}_{\text{OLS}} 
    = \frac{\cov{y, x}}{\dev{x}}
    = \theta + \underbrace{\frac{\cov{\alpha_i, x}}{\dev{x}}}_{\text{质量偏差 (+) }} + \underbrace{\beta \frac{\cov{S_t, x}}{\dev{x}}}_{\text{季节性偏差 (-) }}
\end{align}

根据模型设定, 式 \eqref{eq:raw_ols} 中的偏差项符号为:
\begin{itemize}
    \item $\frac{\cov{\alpha_i, x}}{\dev{x}} > 0$: 忽略商品质量导致的正向偏差;
    \item $\beta \frac{\cov{S_t, x}}{\dev{x}} < 0$: 忽略促销因素导致的负向偏差.
\end{itemize}
通常情况下, 商品异质性带来的正向偏差占主导地位, 导致 $\est{\theta}_{\text{OLS}}$ 被低估, 向 $0$ 收缩.

\subsubsection{去均值回归 (De-meaned)}

引入固定效应模型, 即通过去均值操作令 $\ddot{x}  = x - \avg{x}_i$. 该操作利用 $\alpha_i$ 不随时间变化的特性 ($\ddot{\alpha} = 0$), 直接剔除了商品固定效应. 变换后的模型见式 \eqref{eq:demeaned_demand_f} :
\begin{align}\label{eq:demeaned_demand_f}
    \ddot{y} = \theta x + \beta \ddot{S}_t + \ddot{\varepsilon}
\end{align}

基于此变换进行 OLS 估计, 去除固定效应的价格弹性估计量见式 \eqref{eq:demeaned_ols} : 
\begin{align} \label{eq:demeaned_ols}
    \est{\theta}_{\text{FE}} 
    = \frac{\cov{\ddot{y}, \ddot{x}}}{\dev{\ddot{x}}}
    = \theta + \beta \frac{\cov{\ddot{S_t}, \ddot{x}}}{\dev{\ddot{x}}}
\end{align}

与 Raw OLS 相比, 该方法成功剔除了正向的质量偏差, 仅保留了负向的季节性偏差. 因此, 估计值 $\est{\theta}_{\text{FE}}$ 通常会比 Raw OLS 显著下降 (即弹性绝对值变大), 从而更接近真实值.

\subsubsection{朴素残差回归 (Naive DML)}

朴素 DML 试图通过第一阶段的机器学习模型预测并剔除所有混杂因素, 其估计量形式见式 \eqref{eq:naive_dml} :
\begin{align}\label{eq:naive_dml}
    \est{\theta}_{\text{naive}} = \frac{\cov{\widetilde{y}, \widetilde{x}}}{\dev{\widetilde{x}}}
\end{align}

然而, 在实际应用中, 第一阶段模型往往存在过拟合或因正则化导致的收缩效应, 使得估计出的残差 $\widetilde{x}$ 混入了不可忽略的测量误差 $\nu$. 记观测到的残差形式为 \eqref{eq:measure_error} :
\begin{align}\label{eq:measure_error}
    \widetilde{x} = \widetilde{x}^\star + \nu.
\end{align}

将式 \eqref{eq:measure_error} 代入 \eqref{eq:naive_dml}, 可得:
\begin{align}
    \est{\theta}_{\text{naive}} = \theta \frac{\dev{\widetilde{x}^\star}}{\dev{\widetilde{x}^\star} + \dev{\nu}} =  \theta \times \qty(1 - \frac{\dev{\nu}}{\dev{\widetilde{x}^\star} + \dev{\nu}})
\end{align}

当第一阶段模型预测过于精准时, 真实信号 $\widetilde{x}^\star$ 的方差趋近于 $0$, 导致分母主要由噪音方差 $\dev{\nu}$ 构成. 此时信噪比极低, 导致严重的衰减偏差, 迫使系数 $\est{\theta}_{\text{naive}}$ 向 $0$ 收缩.

\subsubsection{稳健 DML (Robust DML)}

为解决上述问题, 我们采用 Chernozhukov 等人提出的 Neyman 正交化估计量 (Neyman Orthogonal Estimator). 该方法通过修正分母项来构建对滋扰参数误差不敏感的统计量:
\begin{align}
    \label{eq:robust_dml}
    \est{\theta}_{\text{robust}} = \frac{\cov{\widetilde{y}, \widetilde{x}}}{\cov{\widetilde{x}, x}} = \theta \times \frac{\dev{\widetilde{x}^\star}}{\cov{\widetilde{x}^\star + \nu, x}} = \theta \times \frac{\dev{\widetilde{x}^\star}}{\cov{\widetilde{x}^\star, \est{x} + \widetilde{x}^\star}}
\end{align}

其中 $\est{x}$ 为 $x$ 的预测值. 在理想情况下, 真实残差 $\widetilde{x}^\star$ 与预测值正交 ($\cov{\widetilde{x}^\star, \est{x}} = 0$), 因此式 \eqref{eq:robust_dml} 可化简为:
\begin{align} \label{eq:robust_approx}
    \est{\theta}_{\text{robust}} = \theta \times \frac{\dev{\widetilde{x}^\star}}{\dev{\widetilde{x}^\star}} \approx \theta.
\end{align}

式 \eqref{eq:robust_approx} 利用了残差的正交性质, 即便第一阶段模型的收敛速度较慢 (存在估计误差 $\nu$), 第二阶段也能通过协方差结构抵消其影响. 这一改进不仅修正了 Naive DML 的衰减偏差, 同时保留了 DML 处理高维混杂因素的能力, 从而实现对真实弹性的稳健估计.



\section{蒙特卡洛模拟研究}\label{sec:simulation}

为了验证上述理论推导的正确性, 特别是 Robust DML 在存在测量误差下的有效性, 我们构建了一个受控的蒙特卡洛模拟环境. 该环境允许我们在已知真实弹性 $\theta$ 的前提下, 观测不同估计量的表现. 

\subsection{数据生成过程}

设定真实的数据生成过程如下, 以模拟包含高维混杂因素与测量误差的真实市场环境:
\begin{enumerate}
    \item 设定真实价格弹性为 $\theta = -1.0$.
    \item 
    \begin{itemize}
        \item 引入商品固定效应 $\alpha_i \sim N(0, 1)$, 且设定 $\cov{P, \alpha} > 0$ (模拟高质量高价);
        \item 引入时间混杂因素 $S_t \sim \text{Seasonality}$, 且设定 $\cov{P, S} > 0$ (模拟旺季涨价).
    \end{itemize}
    \item 在模拟 DML 第一阶段时, 人为向残差中注入高斯白噪声 $\nu \sim N(0, 0.001^2)$, 以模拟机器学习模型在有限样本下的非完美预测 (即式 \eqref{eq:measure_error} 中的测量误差).
\end{enumerate}

\subsection{模拟结果分析}
\begin{figure}[htbp]
    \centering
    \includegraphics[width=\linewidth]{figs/simulation_verification.pdf}
    \caption{四种估计量的模拟结果对比 (真实值 $\theta = -1.0$)}
    \label{fig:sim_result}
\end{figure}

基于上述 DGP 生成的 $N=5000$ 条观测数据, 四种估计量的分布与偏差如图 \ref{fig:sim_result} 所示. 实验结果与理论推导高度一致:
\begin{enumerate}
    \item 模拟结果显示 $\est{\theta}_{\text{OLS}} \approx -0.5013$, 远高于真实值 $-1.0$. 这验证了式 \eqref{eq:raw_ols}. 由于 $\alpha_i$ 和 $S_t$ 均与价格正相关, 导致了巨大的正向偏差项, 掩盖了真实的需求弹性.
    \item 去均值模型的估计值为 $\est{\theta}_{\text{FE}} \approx -1.6575$. 相比 Raw OLS, 其绝对值显著增大. 这对应了理论分析中 $\alpha_i$ 的消除. 然而, 由于模型仍受时间变动因素 $S_t$ (如式 \eqref{eq:demeaned_ols} 所示) 的影响, 估计结果依然存在偏差.
    \item 在人为注入第一阶段预测噪音 $\nu$ 后, Naive DML 的估计值为 $\est{\theta}_{\text{naive}} \approx -0.7288$. 这一结果精确验证了式 \eqref{eq:measure_error} 中的衰减偏差. 尽管 DML 试图剔除混杂, 但分母中的噪音方差 $\dev{\nu}$ 导致了信噪比下降, 迫使系数向 $0$ 收缩. 且模拟表明, 这种收缩效应在数量级上与遗漏变量偏差相当, 极易误导决策.
    \item 采用 Neyman 正交化公式后, 估计值 $\est{\theta}_{\text{robust}} \approx -1.0095$, 几乎完美还原了真实参数. 这证实了式 \eqref{eq:robust_dml} 的推导. 即便第一阶段残差 $\widetilde{x}$ 包含大量人为注入的噪音, 通过利用原始价格 $x$ , 协方差结构 $\cov{\widetilde{x}^* + \nu, x}$ 成功过滤了噪音干扰, 实现了无偏估计.
\end{enumerate}


\subsection{敏感性分析}

为了进一步验证结论的稳健性, 我们在 $\theta \in [-4.0, 1.0]$ 的范围内进行了敏感性测试 (如图 \ref{fig:sensitivity} 所示).

\begin{figure}[htbp]
    \centering
    \includegraphics[width=\linewidth]{figs/elasticity_sensitivity_analysis.pdf}
    \caption{估计偏差随真实弹性变化的敏感性分析}
    \label{fig:sensitivity}
\end{figure}

结果表明:
\begin{enumerate}
    \item \textbf{Naive DML} 表现出明显的乘性偏差, 即真实弹性绝对值越大, 其绝对误差越大.
    \item \textbf{Robust DML} 在整个定义域内始终紧贴真实弹性值, 证明了该方法不仅能剔除混杂因素, 且对模型预测误差具有极强的鲁棒性.
\end{enumerate}


\section{数据描述与实证设计}\label{sec:dataset}

在前文的理论分析与蒙特卡洛模拟中, 我们验证了 Robust DML 估计量在处理内生性与测量误差方面的优越性. 然而, 相较于受控的模拟环境, 真实世界的零售数据具有更高维的噪声、非平衡的面板结构以及复杂的非线性混杂特征. 为了将理论模型有效地转化为实际定价决策支持, 本节将详细阐述实证研究的数据基础与实施细节. 


\subsection{数据来源与原始分布}

本研究采用 Kaggle 公开数据集 ``\href{https://www.kaggle.com/datasets/aslanahmedov/market-basket-analysis}{Association Rules and Market Basket Analysis}'', 该数据集记录了某在线零售商在特定时期内的真实交易流水. 原始数据是为购物篮分析设计的细粒度流水账, 每一行代表单笔订单中的一项商品记录. 数据总量共计 541,909 条, 涵盖了从交易编号、商品代码(StockCode)、描述信息、交易数量(Quantity)、单价 (UnitPrice) 到客户所在地等核心维度. 原始特征及其具体含义见表 \ref{tab:feature_description}.

\begin{table}[htbp]
    \centering
    \caption{原始数据集特征说明}
    \label{tab:feature_description}
    \begin{tabular}{cl}
        \toprule
        \textbf{特征名称} & \multicolumn{1}{c}{\textbf{含义说明}} \\
        \midrule
        InvoiceNo   & 发票编号: 每笔交易的唯一 6 位编号 \\
        StockCode   & 商品编码: 每种独特商品的唯一 5-6 位字母数字代码 \\
        Description    & 商品描述: 商品的具体名称 \\
        Quantity          & 交易数量: 每笔交易中该类商品的购买件数 \\
        InvoiceDate        & 发票日期: 交易发生的日期和具体时间 \\
        UnitPrice       & 商品单价: 单位商品的销售价格 \\
        CustomerID   & 客户编号: 每名客户的唯一 5 位识别码 \\
        Country     & 国家: 客户居住或订单发生的国家/地区 \\
        \bottomrule
    \end{tabular}
\end{table}


\subsection{数据清洗与聚合策略}

为了构建适用于价格弹性估计的计量模型, 本研究需将分析维度从``单次交易''聚合至``商品-日期''维度, 以构建时间序列上的价格与需求对应关系. 此外, 为了确保因果效应估计的准确性, 本研究实施了严格的数据清洗流程, 总体步骤为:
\begin{enumerate}
    \item 剔除非商品记录: 过滤了 StockCode 中包含 \texttt{['POST', 'DOT', 'M', ...]} 等非交易性编码的记录. 这些记录通常代表邮费、手续费、银行费用或运营调整, 不属于市场供需驱动的商品销售, 若不剔除会干扰价格弹性的计算. 
    \item 处理异常值与筛选相对价格: 针对零售数据中常见的数据录入错误及非理性极值, 本研究计算了每个商品在全观测期内的中位数价格作为基准锚点 ($P_{\text{median}}$). 我们定义相对价格比率 $R_p = P_t / P_{\text{median}}$, 并仅保留 $R_p \in [1/3, 3]$ 区间内的样本. 该阈值设定基于经验法则, 旨在保留正常的商业调价行为(如 3 折促销), 同时剔除系统错误或特殊赠品记录.
    \item 聚合数据: 以``日期 (Date)''、``商品代码 (StockCode)''和``国家 (Country)''为联合主键对数据进行聚合. 
    \begin{itemize}
        \item {销量 ($Q$)}: 采用当日总销售数量 ($\sum \text{Quantity}$), 反映市场总需求.
        \item {价格 ($P$)}: 采用基于销售额加权的平均单价 ($\sum \text{Revenue} / \sum \text{Quantity}$). 相比简单算术平均, 加权价格能更真实地反映当日大多数消费者实际支付的成交价.
    \end{itemize}
\end{enumerate}

清洗与聚合后的数据展现出显著的时间异质性. 如图 \ref{fig:daily_trends} 所示, 每日总销量与交易频次呈现高度的协同波动, 且存在明显的峰谷特征. 这表明市场需求受到宏观时间因素(如节假日、季节)的强烈驱动, 验证了在模型中控制时间混杂因素的必要性.

\begin{figure}[htbp]
    \centering
    \includegraphics[width=\linewidth]{figs/daily_trends_dual_axis.pdf}
    \caption{售出商品数量, 交易笔数与收益时间序列分布图}
    \label{fig:daily_trends}
\end{figure}

\subsection{特征工程与混杂变量构造}

为了解决价格内生性问题, 本研究基于领域知识构建了高维特征空间 $\mathcal{X}$, 旨在从时间、产品生命周期、文本语义及地理四个维度, 捕捉影响供需关系的深层混杂机制.

\subsubsection{非线性时间效应的捕捉}
市场需求具有显著的时间波动规律. 如果忽略这些因素, 可能会将节日带来的“量价齐升”错误地识别为正向的价格弹性.

本研究提取了月份 ({Month})、月内日期 ({Day of Month}) 以及周几 ({Day of Week}). 其中, 月份捕捉了季节性趋势(如冬季对保暖用品的需求增加);月内日期捕捉了发薪日效应(月初购买力较强);周几则捕捉了工作日与周末的消费习惯差异.

通过对这些分类变量进行独热编码, DML 模型的第一阶段能够非线性地拟合出销量的``基准时间趋势'', 从而确保价格残差不再包含由于时间同步性导致的伪相关.

\subsubsection{商品异质性与生命周期控制}

% TODO: 绘制 Dataset 中 Stock Age Days 和 SKU Median Price 的分布
\begin{figure}[htbp]
    \centering
    \includegraphics[width=\linewidth]{figs/feature_distributions.pdf}
    \caption{混杂因素分布:商品生命周期 (左) 与对数化基准价格 (右) 的核密度估计}
    \label{fig:eda_dist}
\end{figure}

不同品类的商品具有不同的基准价格和需求分布, 且同一商品在不同生命周期的价格敏感度各异. 我们构造了两个关键连续变量, 并对其分布进行了核密度估计 (如图 \ref{fig:eda_dist} 所示):
\begin{itemize}
    \item 商品生命周期 (Stock Age Days): 定义为当前交易日期与该商品首次进入系统日期之差.  如图 \ref{fig:eda_dist} (左) 所示, 数据覆盖了从``新品引入期'' (左侧峰值) 到``成熟期/衰退期'' (右侧拖尾) 的完整周期. 新品通常享有流量红利, 而衰退期商品常伴随清仓甩卖 (低价高销). 若不控制此变量, 模型会将生命周期带来的自然销量波动混淆为价格弹性.

    \item 基准锚点价 (SKU Median Price): 定义为每种商品在历史观测期内的单价中位数. 该指标作为商品“档次”或“品质”的代理变量, 可以捕捉不同价格带商品的固有基准销量差异, 从而隔离了商品异质性产生的截距偏差. 如图 \ref{fig:eda_dist} (右) 所示, 价格分布呈现典型的右偏长尾特征 (Log-Normal 分布). 这表明市场中存在少量高价商品. 为防止数值问题影响模型收敛, 本研究后续对该变量进行了标准化 (StandardScaler) 处理.
\end{itemize}


\subsubsection{基于自然语言处理的细粒度属性挖掘}
原始数据集中的 Description 字段包含了丰富的非结构化信息, 这些信息往往定义了商品的细分市场. 本研究使用 N-gram 词频向量化: 利用 CountVectorizer 对商品描述进行文本挖掘, 提取一元至三元短语(1-3 grams), 并设定最小文档频率限制(min\_df=0.0025)以剔除长尾低频词. 

该方法能自动识别如 ``SILK'' (材质)、``VINTAGE'' (风格)、``SET OF 6'' (规格) 等关键属性. 例如, 大规格包装 (``SET'') 通常单位价格较低但需求量稳定. 将这些文本特征纳入 DML 的控制变量, 有助于模型在更细粒度的属性组合上平衡比较组, 从而更精准地隔离价格效应.

\subsubsection{地域固定效应}

考虑到不同国家(如英国、法国、德国)的消费水平、物流成本及节假日安排差异, 本研究将 Country 作为分类控制变量. 这有助于消除地理因素产生的系统性误差, 例如英国市场的价格调整策略可能与欧洲大陆市场完全不同. 

\subsubsection{连续变量的标准化处理}

为了提高机器学习模型(如岭回归及随机森林)的收敛速度和正则化效率, 本研究对所有连续型控制变量(如在架时长、中位数价格)进行了标准正态化处理 (StandardScaler). 这一步确保了不同量纲的特征在 DML 模型中具有公平的贡献度, 防止大数值特征(如天数)掩盖小数值特征(如标准化后的价格波动)的信号. 


\subsection{实证模型设计}

基于上述构建的高维特征空间 $\mathcal{X}$, 本研究针对零售数据的分布特性及微观交易数据的极高噪声, 设计了如下混合残差与分箱推断策略\cite{schultz2024causalforecastingpricing}:

\begin{enumerate}
    \item {价格模型}: 采用{随机森林回归} 估计 $g(\mathcal{X}) = \mathbb{E}[P|\mathcal{X}]$. 价格制定机制往往是非线性的 (如由季节、库存、竞品共同决定的复杂规则), 随机森林能有效捕捉高维特征间的交互作用, 从而获得高质量的价格残差.
    
    \item {销量模型}: 采用{Poisson 回归} 估计 $m(\mathcal{X}) = \mathbb{E}[Q|\mathcal{X}]$. 销量本质上是取值为非负整数的计数数据 (Count Data). 相比传统线性回归, Poisson 回归能更准确地拟合长尾分布, 避免预测出负销量的不合理现象.
    
    \item {分箱残差回归}: 在获得正交化残差 $\widetilde{P}$ 和 $\widetilde{Q}$ 后, 本研究摒弃了传统的对所有残差样本进行直接线性回归的做法, 而是采用{分箱最小二乘法} 作为计算价格弹性的核心算法. 具体步骤如下:
    \begin{itemize}
        \item {分箱}: 将价格残差 $\widetilde{P}$ 依据分位数划分为 $K$ 个等频区间 (本研究设定 $K=15$).
        \item {聚合}: 计算每个区间内 $\widetilde{P}$ 和 $\widetilde{Q}$ 的均值点, 构造 $K$ 个代表性样本点. 这一过程本质上是在进行非参数化的局部平均平滑 (Local Avaraging), 能够有效抵消微观层面的随机测量误差.
        \item {斜率估计}: 对这 $K$ 个聚合均值点进行加权最小二乘回归, 其回归系数即为最终报告的价格弹性 $\est{\theta}$.
    \end{itemize}
    该策略不仅能直观地可视化需求曲线的形态, 更能在有限样本下提供比传统 OLS 更稳健的点估计.
\end{enumerate}



\section{实证结果分析} \label{sec:evidence}

基于前文构建的混合残差双重机器学习模型, 本节对 Kaggle 零售数据集的价格弹性进行估算. 我们通过对比不同计量模型的估计结果, 并结合可视化诊断与误差分析, 验证 Robust DML 框架在处理内生性问题上的有效性.

\subsection{价格弹性估计结果对比}

表 \ref{tab:elasticity_results} 详细汇总了不同估计策略下的价格弹性系数 ($\est{\theta}$) 与拟合优度指标. 

\begin{table}[htbp]
    \centering
    \caption{不同模型设定下的价格弹性估计与误差分析}
    \label{tab:elasticity_results}
    \begin{tabular}{c c c c}
        \toprule
        \textbf{估计策略} & \textbf{弹性系数 ($\est{\theta}$)} & \textbf{Binned RMSE} & \textbf{结果诊断} \\
        \midrule
        Raw OLS & -0.607 & 0.308 & 严重低估, 供需联立偏差 \\
        De-meaned (FE) & -1.803 & 0.119 & 显著增大, 剔除固定效应 \\
        Naive DML & -0.580 & 0.108 & 衰减偏差, 第一阶段噪音干扰 \\
        Robust DML & -1.051 & 0.337 & 因果修正, Neyman 正交化 \\
        \bottomrule
    \end{tabular}
\end{table}

随着模型处理阶段的深入, RMSE 总体呈现下降趋势. 值得注意的是, 尽管 Robust DML 的 Binned RMSE 高于 Naive DML, 但这并不意味着模型失效, 其差异源于两者优化目标的本质不同:
\begin{itemize}
    \item Naive DML (OLS) 的数学目标就是最小化残差平方和, 在包含噪音的样本中, OLS 倾向于过度拟合这些随机扰动以降低误差,从而导致参数估计产生衰减偏差.
    \item Robust DML 的目标是因果识别, 即利用原始价格修正分母偏差. 它利用原始价格信息构建 Neyman 正交化分数以修正分母. 这一过程虽然牺牲了对当前样本特定噪音的拟合精度, 导致 RMSE 上升,却换取了参数估计的无偏性与一致性. 
\end{itemize}

\subsection{需求曲线剖析}

\begin{figure}[htbp]
    \centering
    \includegraphics[width=\linewidth]{figs/demand_curve_comparison_robust.pdf}
    \caption{需求曲线重构: 原始数据 (Raw)、去均值 (De-meaned)、朴素 DML (Naive) 与稳健 DML (Robust)}
    \label{fig:demand_curve_robust}
\end{figure}

为了直观展示因果推断框架剥离混杂因素的效果, 图 \ref{fig:demand_curve_robust} 绘制了不同估计策略下的需求曲线. 结合表 \ref{tab:elasticity_results} 的实证数据, 我们得出以下结论:

\begin{itemize}
    \item {Raw OLS 的低估现象 (灰色虚线)}: 
    原始数据的回归斜率平缓 ($\est{\theta} \approx -0.61$). 这直观地反映了``供需联立性偏差'': 旺季的高需求往往伴随着商家的维持高价策略, 这种正相关力量抵消了真实的价格负效应, 导致模型低估了用户对价格的敏感度.
    
    \item {De-meaned 的过度敏感 (蓝色虚线)}: 
    在剔除商品固定效应后, 斜率显著变陡 ($\est{\theta} \approx -1.80$). 这表明商品本身的异质性 (如高档商品销量低) 是主要的混杂来源. 然而, 简单的去均值无法处理随时间变化的混杂因素 (如全场大促), 可能将促销带来的自然流量误归因为降价效应, 从而一定程度上高估了弹性.
    
    \item {Naive DML 的衰减偏差 (红色虚线)}: 
    尽管引入了双重机器学习, 但直接回归残差得到的弹性系数却回落至 -0.58. 这与第 \ref{sec:simulation} 节模拟实验的结论高度一致: 由于第一阶段模型过度拟合了噪音, 导致价格残差的方差收缩, 引发了严重的衰减偏差, 使得估计值向 0 偏移.
    
    \item {Robust DML 的因果修正 (绿色虚线)}: 
    采用 Neyman 正交化公式修正后, 弹性系数被修正为 -1.05. 
    该结果介于 Raw OLS 与 De-meaned 之间, 具有最高的理论可信度. 一方面, 它像 De-meaned 一样剔除了商品固定效应; 另一方面, 它通过第一阶段的随机森林控制了时间与文本特征, 修正了 De-meaned 模型因忽略季节性因素而导致的高估偏差. 最终 -1.05 的弹性系数表明, 该市场呈现接近单位弹性的特征.
\end{itemize}

\subsection{稳健性检验}

为了进一步验证前文估计结果的可靠性, 使用 $2$ 折交叉拟合策略: 利用样本 A 训练辅助模型并预测样本 B 的残差 (反之亦然), 从而彻底消除过拟合带来的自身偏差,






\end{document}