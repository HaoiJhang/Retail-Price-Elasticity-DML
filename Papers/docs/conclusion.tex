\section{研究结论与展望} \label{sec:conclusion}

\subsection{研究总结}


针对零售观测数据中普遍存在的内生性与高维混杂难题, 本研究构建了基于 Neyman 正交化的双重机器学习 (Robust DML) 框架. 通过蒙特卡洛模拟与实证分析, 我们得出以下核心结论:
\begin{enumerate}
    \item \textbf{方法论有效性}: 传统 OLS 模型因忽略供需联立性产生正向偏差, 而去均值模型 (Fixed Effects) 因忽略时变混杂产生过度敏感偏差. Robust DML 有效修正了上述两类偏差, 并在有限样本下克服了朴素 DML 的衰减偏差. 
    \item \textbf{市场弹性特征}: 实证结果显示, 该在线零售市场的平均价格弹性为 \textbf{-1.05}. 这表明市场呈现典型的单位弹性特征, 即总体而言, 价格变动幅度与需求响应幅度基本持平, 总营收对价格调整表现出较强的刚性. 
\end{enumerate}

\subsection{管理启示}

基于 $\est{\theta} \approx -1.05$ 的估计结果, 我们为零售厂商提出以下定价策略建议:
\begin{itemize}
    \item \textbf{营收视角}: 由于市场接近单位弹性, 单纯的全面降价或提价策略在短期内难以显著改变总营收 (Revenue Neutral). 企业应将关注点从``价格战''转移至提升服务质量、优化库存周转等非价格竞争维度. 
    \item \textbf{市场份额视角}: 尽管营收变动不大, 但 $-1.05$ 的弹性意味着降价仍能带来销量的超比例增长. 若企业的战略目标是快速清库存或抢占市场份额 (而非短期利润最大化), 激进的促销策略依然有效. 
\end{itemize}

\subsection{未来展望:基于品类结构的异质性弹性模型}

本研究目前的实证结果主要反映了市场的平均价格弹性 (Average Treatment Effect, ATE). 然而, 正如稳健性检验所提到的, 在实际零售场景中, 不同层级品类 的商品往往具有显著的异质性. 例如, 奢侈品、耐用品与日用快消品的消费者决策机制截然不同, 使用单一的全局弹性系数可能掩盖了细分市场的结构性机会. 

为了捕捉这种结构性差异, 未来的研究可参考双对数结构嵌套均值模型, 将弹性系数参数化为品类特征的函数. 假设价格弹性 $\theta$ 并非由常数决定, 而是由品类指示向量 $\vb{L}_i$ (One-hot Encoding) 动态调节:
\begin{align} \label{eq:hte_model}
    \theta_i = \theta_{\text{base}} + \vb*{\delta}^\top \vb{L}_i
\end{align}

其中, $\theta_{\text{base}}$ 为基准弹性, $\vb*{\delta}$ 为不同品类相对于基准的弹性偏移量 (Effect Modifiers). 在 DML 的第二阶段推断中, 我们可以将正交化后的残差回归方程扩展为包含交互项的形式:
\begin{align}
    \widetilde{y}_i = (\theta_{\text{base}} + \vb*{\delta}^\top \vb{L}_i) \cdot \widetilde{x}_i + \varepsilon_i = \theta_{\text{base}} \widetilde{x}_i + \vb*{\delta}^\top (\vb{L}_i \cdot \widetilde{x}_i) + \varepsilon_i
\end{align}

通过对上述交互项模型进行回归, 我们不仅能获得全局基准弹性, 还能同时识别出不同细分品类的价格敏感度差异 (即条件平均处理效应, CATE). 这将有助于企业从``一刀切''的定价策略转向精细化、差异化的品类定价管理, 从而在微观层面实现利润最大化. 