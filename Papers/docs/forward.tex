
前文所述的 Robust DML 估计量 $\est{\theta}$ 反映了市场的平均价格敏感度 (Average Treatment Effect). 然而,在实际零售场景中,不同层级品类 (Category) 的商品往往具有显著的异质性。例如,奢侈品与日用品的弹性截然不同。

为了捕捉这种结构性差异,参考双对数结构嵌套均值模型 (Double-log Structural Nested Mean Model),我们将弹性系数参数化为品类特征的函数。假设价格弹性 $\theta$ 并非由常数决定,而是由品类指示向量 $\mathbf{L}_i$ (One-hot Encoding) 调节:

\begin{align} \label{eq:hte_model}
    \theta_i = \theta_{base} + \boldsymbol{\delta}^\top \mathbf{L}_i
\end{align}

其中,$\theta_{base}$ 为基准弹性,$\boldsymbol{\delta}$ 为不同品类相对于基准的弹性偏移量。在 DML 的第二阶段推断中,我们将正交化后的残差回归方程扩展为包含交互项的形式:

\begin{align}
    \widetilde{y}_i = (\theta_{base} + \boldsymbol{\delta}^\top \mathbf{L}_i) \cdot \widetilde{x}_i + \epsilon_i = \theta_{base} \widetilde{x}_i + \boldsymbol{\delta}^\top (\mathbf{L}_i \cdot \widetilde{x}_i) + \epsilon_i
\end{align}

通过对上述交互项模型进行回归,我们不仅能获得全局弹性,还能同时识别出不同细分品类的价格敏感度差异,从而为精细化定价提供理论支撑。
