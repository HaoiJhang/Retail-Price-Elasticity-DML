
\section{实证结果分析} \label{sec:evidence}

基于前文构建的混合残差双重机器学习模型, 本节对 Kaggle 零售数据集的价格弹性进行估算. 我们通过对比不同计量模型的估计结果, 并结合可视化诊断与误差分析, 验证 Robust DML 框架在处理内生性问题上的有效性.

\subsection{价格弹性估计结果对比}

表 \ref{tab:elasticity_results} 详细汇总了不同估计策略下的价格弹性系数 ($\est{\theta}$) 与拟合优度指标. 

\begin{table}[htbp]
    \centering
    \caption{不同模型设定下的价格弹性估计与误差分析}
    \label{tab:elasticity_results}
    \begin{tabular}{c c c c}
        \toprule
        \textbf{估计策略} & \textbf{弹性系数 ($\est{\theta}$)} & \textbf{Binned RMSE} & \textbf{结果诊断} \\
        \midrule
        Raw OLS & -0.607 & 0.308 & 严重低估, 供需联立偏差 \\
        De-meaned (FE) & -1.803 & 0.119 & 显著增大, 剔除固定效应 \\
        Naive DML & -0.580 & 0.108 & 衰减偏差, 第一阶段噪音干扰 \\
        Robust DML & -1.051 & 0.337 & 因果修正, Neyman 正交化 \\
        \bottomrule
    \end{tabular}
\end{table}

随着模型处理阶段的深入, RMSE 总体呈现下降趋势. 值得注意的是, 尽管 Robust DML 的 Binned RMSE 高于 Naive DML, 但这并不意味着模型失效, 其差异源于两者优化目标的本质不同:
\begin{itemize}
    \item Naive DML (OLS) 的数学目标就是最小化残差平方和, 在包含噪音的样本中, OLS 倾向于过度拟合这些随机扰动以降低误差,从而导致参数估计产生衰减偏差.
    \item Robust DML 的目标是因果识别, 即利用原始价格修正分母偏差. 它利用原始价格信息构建 Neyman 正交化分数以修正分母. 这一过程虽然牺牲了对当前样本特定噪音的拟合精度, 导致 RMSE 上升,却换取了参数估计的无偏性与一致性. 
\end{itemize}

\subsection{需求曲线剖析}

\begin{figure}[htbp]
    \centering
    \includegraphics[width=\linewidth]{figs/demand_curve_comparison_robust.pdf}
    \caption{需求曲线重构: 原始数据 (Raw)、去均值 (De-meaned)、朴素 DML (Naive) 与稳健 DML (Robust)}
    \label{fig:demand_curve_robust}
\end{figure}

为了直观展示因果推断框架剥离混杂因素的效果, 图 \ref{fig:demand_curve_robust} 绘制了不同估计策略下的需求曲线. 结合表 \ref{tab:elasticity_results} 的实证数据, 我们得出以下结论:

\begin{itemize}
    \item {Raw OLS 的低估现象 (灰色虚线)}: 
    原始数据的回归斜率平缓 ($\est{\theta} \approx -0.61$). 这直观地反映了``供需联立性偏差'': 旺季的高需求往往伴随着商家的维持高价策略, 这种正相关力量抵消了真实的价格负效应, 导致模型低估了用户对价格的敏感度.
    
    \item {De-meaned 的过度敏感 (蓝色虚线)}: 
    在剔除商品固定效应后, 斜率显著变陡 ($\est{\theta} \approx -1.80$). 这表明商品本身的异质性 (如高档商品销量低) 是主要的混杂来源. 然而, 简单的去均值无法处理随时间变化的混杂因素 (如全场大促), 可能将促销带来的自然流量误归因为降价效应, 从而一定程度上高估了弹性.
    
    \item {Naive DML 的衰减偏差 (红色虚线)}: 
    尽管引入了双重机器学习, 但直接回归残差得到的弹性系数却回落至 -0.58. 这与第 \ref{sec:simulation} 节模拟实验的结论高度一致: 由于第一阶段模型过度拟合了噪音, 导致价格残差的方差收缩, 引发了严重的衰减偏差, 使得估计值向 0 偏移.
    
    \item {Robust DML 的因果修正 (绿色虚线)}: 
    采用 Neyman 正交化公式修正后, 弹性系数被修正为 -1.05. 
    该结果介于 Raw OLS 与 De-meaned 之间, 具有最高的理论可信度. 一方面, 它像 De-meaned 一样剔除了商品固定效应; 另一方面, 它通过第一阶段的随机森林控制了时间与文本特征, 修正了 De-meaned 模型因忽略季节性因素而导致的高估偏差. 最终 -1.05 的弹性系数表明, 该市场呈现接近单位弹性的特征.
\end{itemize}

\subsection{基于交叉拟合的稳健性检验}

在利用机器学习算法估计滋扰参数时,若在同一样本上同时进行模型训练与残差预测,极易产生过拟合导致的``自身观测偏差''. 为消除这一系统性误差,本研究采用了 $2$ 折交叉拟合策略对估计结果进行稳健性检验.

具体的实施步骤如下:
\begin{enumerate}
    \item \textbf{样本分割}: 将全样本索引集 $I = \{1, \dots, N\}$ 随机等分为两个不重叠的子集 $I_A$ 和 $I_B$, 使得 $I_A \cup I_B = I$ 且 $I_A \cap I_B = \emptyset$.
    \item \textbf{交叉预测}: 
    \begin{itemize}
        \item 利用子样本 $I_A$ 训练价格模型 $\est{g}_A(X)$ 与销量模型 $\est{m}_A(X)$, 并对子样本 $I_B$ 中的观测值进行预测, 计算残差 $\widetilde{P}_{B}$ 与 $\widetilde{Q}_{B}$.
        \item 同理, 利用子样本 $I_B$ 训练辅助模型, 对子样本 $I_A$ 进行外样本预测 (Out-of-Sample Prediction), 获取残差 $\widetilde{P}_{A}$ 与 $\widetilde{Q}_{A}$.
    \end{itemize}
    \item \textbf{全局估计}: 将两部分的残差合并, 构建正交化后的全样本残差集, 并在最后一步基于 Neyman 正交化公式计算最终的弹性系数 $\est{\theta}$.
\end{enumerate}

通过上述过程, 每一个样本点的残差均是由未包含该样本的模型预测得到的. 这种机制有效地切断了预测误差与模型估计之间的相关性. 

\begin{figure}[htbp]
    \centering
    \includegraphics[width=\linewidth]{figs/dml_diagnostics_final.pdf}
    \caption{交叉拟合残差: 不同颜色代表不同折的外样本残差分布}
    \label{fig:cross_fitting_diag}
\end{figure}

图 \ref{fig:cross_fitting_diag} 直观展示了交叉拟合后的残差分布情况. 观察发现, 不同颜色 (不同折) 的残差分布高度重合, 未出现结构性分离, 证明了模型的泛化能力. 值得注意的是,基于交叉拟合的 Robust DML 估计结果与去均值固定效应模型的估计结果非常接近.

这一现象揭示了该数据集的内在因果结构:商品层面的异质性是导致内生性偏差的主导因素. 传统的固定效应模型通过剔除截距偏差,已经消除了绝大部分混杂影响. 

然而, DML 的价值在于其提供了更严格的理论保障. 它进一步证实了,在控制了非线性时间趋势和文本特征后,价格弹性的点估计依然维持在稳定区间. 这不仅验证了线性固定效应模型在本场景下的适用性,也排除了潜在的非线性混杂因素对结论的重大干扰.