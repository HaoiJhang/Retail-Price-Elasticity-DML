\section{引言}

在微观经济学与现代商业决策中, 需求价格弹性是衡量市场反应的核心指标. 它反映了消费者需求量对价格变动的敏感程度, 直接决定了企业的定价策略与盈利能力. 从管理决策的角度来看, 准确估计价格弹性具有重要的指导意义: 当弹性绝对值小于 $1$ 时, 适度提价能够增加总收入; 而当弹性绝对值大于 $1$ 时, 降价促销则成为获取市场份额、提升营收的有效手段\citep{nicholson2011intermediate}. 因此, 在动态竞争的市场环境中, 如何精准捕捉需求曲线的斜率, 成为厂商实现收益最大化的关键.

尽管价格弹性在理论上定义明确, 但在实证估算中却面临严峻挑战. 理想的评估手段是进行随机对照试验, 即通过随机向不同用户展示差异化价格来观察反馈. 然而在现实商业环境中, 这种定价策略往往由于可能损害用户体验、导致价格歧视争议及削弱品牌信誉而难以大规模推行. 

因此, 基于历史观测数据进行因果推断成为更为可行的替代方案. 然而, 观测数据并非来自随机分配, 价格与需求之间往往存在复杂的内生性问题. 例如, 季节性波动、促销活动的周期性、以及产品质量的变化等因素, 既会影响企业的定价行为, 也会直接干扰消费者的购买决策. 如果不能有效剥离这些混杂因素的影响, 传统的统计回归模型往往会产生严重偏差, 得出虚假的相关性而非真实的因果效应. 

针对上述难题, 本研究引入了前沿的双重机器学习框架\citep{hua2021markdownsecommercefreshretail}, 旨在从高维、非线性的历史交易数据中提取无偏的价格弹性估值. 相比于传统的计量经济学模型, 本方法在以下两个方面具有显著优势: 
\begin{itemize}
    \item 高维变量的处理能力:  本研究利用正则化技术与特征工程, 从商品代码、日期特征、文本描述及区域分布等海量信息中自动筛选重要控制变量, 有效解决了因变量过多导致的过拟合问题. 
    \item 非线性因果建模:  传统的线性模型难以捕捉价格形成的复杂机制. 本研究在 DML 框架下, 结合了随机森林捕捉非线性交互的能力, 以及 Poisson 回归处理离散销量数据的统计优势, 实现了更为精准的预测与正交化处理. 
\end{itemize}

本文利用 Kaggle 公开的真实零售交易数据集进行实证分析. 实验结果表明, 通过 DML 框架提取的正交化残差能够更清晰地还原需求曲线的线性结构, 显著降低了估计误差, 并为企业在复杂环境下进行``智能定价''提供了稳健的量化支持. 

本文的后续章节结构安排如下:

第 \ref{sec:theory} 节构建了需求价格弹性的理论计量框架,通过数学推导深入剖析了普通最小二乘法 (OLS)、去均值模型 (De-meaned) 及朴素双重机器学习 (Naive DML) 在因果推断中产生偏差的内在机制,并从理论上论证了 Robust DML 估计量的无偏性.

第 \ref{sec:simulation} 节设计了受控的蒙特卡洛模拟实验,在已知真实弹性参数与人为注入噪音的前提下,验证了上述理论推导的正确性,并评估了不同估计量在有限样本下的稳健性.

第 \ref{sec:dataset} 节介绍了实证研究的数据基础与模型设定. 本节详细阐述了 Kaggle 零售数据的预处理流程、基于 NLP 与时间序列的高维特征工程,以及``随机森林 + Poisson 回归''混合模型的具体实施细节. 

第 \ref{sec:evidence} 节展示了实证估算结果与模型诊断,通过分箱回归图重构了需求曲线,并对弹性系数进行了深度分析. 最后,第 \ref{sec:conclusion} 节总结全文, 提出了管理启示与未来展望. 